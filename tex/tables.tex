\documentclass[a4paper,11pt, titlepage]{scrartcl}
\usepackage[utf8]{inputenc}
\usepackage[german]{babel}
\usepackage{amsmath}
\usepackage{textcomp}

\usepackage{pdflscape}
\usepackage[clockwise]{rotating}
%\usepackage{lscape}
\usepackage[left=2.0cm,top=2.0cm,bottom=2.4cm,right=2.0cm]{geometry}

\usepackage{graphicx}
\usepackage{caption}
\usepackage[euler]{textgreek}
\usepackage{wasysym}

\usepackage{longtable, booktabs, tabu}
\usepackage[table]{xcolor}
\definecolor{light-gray}{gray}{0.95}
\newcommand{\wrongCell}{\cellcolor{light-gray}}  %{0.9}

\usepackage{multirow}

\usepackage[hidelinks]{hyperref}

\hypersetup{
       pdfauthor = {Richard Stiller, richard.nao.htwk@gmail.com},
       pdftitle = {Ultraschall-Messungen zum Test \glqq $\jobname$\grqq},
       pdfsubject = {ultraschall messung $\jobname$ },
       pdfkeywords = {ultraschall, hinderniserkennung, Roboterfußball, %
                      NAO, $\jobname$, messung},
       pdfcreator = {LaTeX with hyperref package},
       pdfproducer = {pdflatex}
}

\begin{document}

\titlehead{
\begin{center}
    Hochschule für Technik, Wirtschaft und Kultur Leipzig \\
    Fakultät Informatik, Mathematik und Naturwissenschaften \\
    \vspace{1.6cm} \includegraphics[width=0.5\textwidth]{../img/org/htwk-logo.png}     
\end{center}
}

\subject{Messergebnisse}

\author{Richard Stiller\\
  \small{\texttt{richard.nao.htwk@gmail.com}}}
\title{Ultraschall-Messungen zum Test \glqq $\jobname$\grqq}
\date{\small{\textit{\today}}}


\maketitle
\listoftables
\clearpage
\section*{Tabellen}
\input \jobname.generate.tex
\clearpage
\section*{Anmerkung}   
Alle Messwerte haben die Einheit Meter.
\begin{description}
    \item[Datei] Der Dateiname der Rawdatei mit den Messwerten.
    \item[gefiltert] (optional) Messungen (zeilenweise Erfassung von Messdaten), die keine Hindernis enthalten haben (alle Werte sind \textit{2.55}), wurden nicht abgebildet.
    \item[n] Für die Modi $< 4$ wichtig (\textit{Device/SubDeviceList/US/Sensor/Value}). Dies Werte sind immer identisch mit \textit{r0}.
    \item[l0 .. l9] Bis zu 10 Messwerte (0 bis 9) auf der linken (l) Seite. Beispiel: Messung 1 in Links ist \textit{l1} oder \textit{Device/SubDeviceList/US/Left/Sensor/Value1}.
    \item[r0 .. r9] Bis zu 10 Messwerte (0 bis 9) auf der rechten (r) Seite. Beispiel: Messung 1 in Rechts ist \textit{r1} oder \textit{Device/SubDeviceList/US/Right/Sensor/Value1}.
    \item[Sichtbar] Sagt an, ob das Programm dieses Hindernis erfassen würde, beziehend auf den $[$\textbf{Winkel}$]$ aus der Analyse.
    \item[Soll] Den Wert, den das Programm messen würde, siehe Analyse.
    \item[\diameter{}Ist] Der Durchschnittswert der Messdaten über alle Messungen im jeweiligem Modus von den Spalten \textit{r0} oder \textit{l0}.
    %Der Durchschnittswert der Messdaten über alle abweichenden Messungen.
    \item[Differenz] Differenz zwischen Soll- und Ist-Wert.
    \item[Analyse {$[$Winkel$]$}] (wenn vorhanden) Daten werden mit dem Programm USTest abgeglichen, ob die Werte in einem idealen Kegel mit dem Winkel \textit{Winkel} erfasst sind. Es werden ggf. die Abweichungen der Messwerte ausgegeben. 
    \item[Keine Hindernisse gemessen] Es wurden alle Zeilen eines Durchlaufs heraus gefiltert.
    \item[Keine Analyse] Folgt bei Analyse, wenn \textit{Keine Hindernisse gemessen} auftritt.

\end{description}

\end{document}
